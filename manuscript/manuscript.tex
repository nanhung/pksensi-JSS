\documentclass[article]{jss}
\usepackage[utf8]{inputenc}

\providecommand{\tightlist}{%
  \setlength{\itemsep}{0pt}\setlength{\parskip}{0pt}}

\author{
Nan-Hung Hsieh\\Texas A\&M University \And Brad Reisfeld\\Colorado State University \And Weihsueh A. Chiu\\Texas A\&M University
}
\title{\pkg{pksensi}: An R package to apply sensitivity analysis in
pharmacokinetic models}

\Plainauthor{Nan-Hung Hsieh, Brad Reisfeld, Weihsueh A. Chiu}
\Plaintitle{pksensi: An R package an R package to apply sensitivity analysis in
pharmacokinetic models}
\Shorttitle{\textbf{pksensi}: Sensitivity analysis for pharmacokinetic models in R}

\Abstract{
Sensitivity analysis is an essential tool for modelers to understand the
influence of model parameters on model outputs. It is also increasingly
used in developing and assessing pharmacokinetic models. In our previous
work, we applied a global sensitivity analysis workflow to reduce the
computational burden in the Bayesian Markov Chain Monte Carlo-based
calibration process of a physiologically based pharmacokinetic (PBPK)
model. Although several sensitivity analysis algorithms are available,
no comprehensive package exists that allows users to seamlessly solve
the PBPK model differential equations, run sensitivity analyses,
visualize sensitivity analysis results, and discriminate between the
``non-influential'' model parameters that can be fixed and those that
need calibration. Therefore, we developed an open-source R package,
called \textbf{pksensi}, to fill this need and make sensitivity analysis
more accessible in pharmacological and toxicological research. This
package can investigate both parameter uncertainty and sensitivity in
pharmacokinetic models, such as PBPK model with multivariate model
outputs (multiple time-points and tissue compartments). It also refined
the extended Fourier Amplitude Sensitivity Test method for sensitivity
analysis by adding a random phase-shift to analyze the statistical
variability of the sensitivity index for each model parameter.
Furthermore, pksensi includes functions to check the convergence and
sensitivity of model parameters, providing a means to assess the
robustness of the sensitivity measurement. Utilizing pksensi, we
successfully reproduced our previously published sensitivity analysis
results of human PBPK modeling for acetaminophen and its two primary
metabolites. Overall, pksensi improves the user experience of performing
sensitivity analysis and can create robust and reproducible results for
decision making in pharmacokinetic model calibration.
}

\Keywords{Pharmacokinetics, Physiologically based pharmacokinetics, Sensitivity analysis}
\Plainkeywords{Pharmacokinetics, Physiologically based pharmacokinetics, Sensitivity analysis}

%% publication information
%% \Volume{50}
%% \Issue{9}
%% \Month{June}
%% \Year{2012}
%% \Submitdate{}
%% \Acceptdate{2012-06-04}

\Address{
        Weihsueh A. Chiu\\
  Texas A\&M University\\
  4458 TAMU, College Station, TX 77843, USA\\
  E-mail: \email{wchiu@cvm.tamu.edu}\\
  
  }

% Pandoc header

\usepackage{amsmath}

\begin{document}

\hypertarget{introduction}{%
\section{Introduction}\label{introduction}}

Sensitivity analysis is a mathematical technique to investigate how
variations in model parameters affect model outputs. An increasing
number of studies use sensitivity analysis to determine which model
parameters contribute to high variation in model predictions (Ferretti
et al., 2016). Also, this technique has been applied in pharmacology and
toxicology research (Loizou et al., 2015; McNally et al., 2012;
Scherholz et al., 2018). Pharmacokinetic modeling describes the changes
in the concentrations or amounts of a substance within model
compartments over time. The goal of sensitivity analysis in
pharmacokinetic research is to examine the sensitivity of output
variables (e.g.~compound concentration in blood or tissues) in
pharmacokinetic models responds to input parameters, such as anatomical,
physiological, and kinetic constants (McNally et al., 2011). It can be
further applied to parameter prioritization and parameter fixing before
model calibration (Hsieh et al., 2018).

In our previous work (Hsieh et al., 2018), we developed an approach to
apply global sensitivity analysis workflow to reduce the computational
burden in the Bayesian, Markov Chain Monte Carlo (MCMC)-based
calibration process of a physiologically based pharmacokinetic (PBPK)
model. We used GNU MCSim (Bois, 2009), an effective simulation package
for Bayesian population PBPK modeling, to calibrate the model. We found
that the extended Fourier Amplitude Sensitivity Test (eFAST), a type of
global sensitivity analysis algorithm, had the best balance of
efficiency and accuracy for a complex, multi-compartment, multi-dataset,
and multi-metabolite PBPK model. Also, we found some efficient
visualization approaches that can be used to distinguish between
``influential'' and ``non-influential'' parameters. We also found a
useful approach for communicating the parameter sensitivity in decision
making.

Most of sensitivity analysis tools are constructed using scientific
computing software, such as GNU Octave/Matlab (Pianosi et al., 2015),
python (Herman and Usher, 2017), and R (Ghanem et al., 2017).
Previously, we mainly used R with the sensitivity package (Pujol et al.,
2017), which is a practical tool to conduct local and global sensitivity
analysis. The sensitivity package includes some functions to generate
the parameter sequences by using different computing algorithms. Also,
it provides a feasible way to integrate external modeling results. This
computational approach can effectively solve the heavy computing burden
of numerical solutions within the pure R environment. Several R
packages, such as FME, multisensi, and ODEsensitivity include
sensitivity analysis tools and are freely available in the Comprehensive
R Archive Network. The FME package contains a basic method for
performing local sensitivity analysis for dynamic models (Soetaert and
Petzoldt, 2016). The multisensi package provides sensitivity analysis
for models with multivariate output (Bidot et al., 2018). ODEsensitivity
can perform sensitivity analysis in ordinary differential equation (ODE)
models (Weber et al., 2018). It uses the ODE interface from the deSolve
package, connecting it with the sensitivity analysis from the
sensitivity package, and can perform sensitivity analysis on models with
multivariate output. Although these tools provide various approaches to
perform sensitivity analysis, we have not find a suitable package that
provides functions with which users can visualize and distinguish the
``non-influential'' model parameters and can further apply to parameter
fixing in pharmacokinetic modeling.

Here we develop an R package, called pksensi, which is designed to make
sensitivity analysis more accessible and reproducible in pharmacological
and toxicological researches. This package can investigate both
parameter uncertainty and sensitivity in pharmacokinetic models, such as
PBPK, and advanced compartment absorption and transit models with
multivariate model output. The design concepts of pksensi are:

\begin{itemize}
\tightlist
\item
  Cross-platform: Models can run on Windows/MacOS/Linux
\item
  Freely available: All related packages are free and open source
\item
  Integrated application: Users can run pharmacokinetic models in R with
  script that were written in C or GNU MCSim
\item
  Decision making: The output results and visualization tools can be
  used to easily determine which parameters have ``non-influential''
  effects on the model output and can be fixed in model calibration.
\end{itemize}

\hypertarget{workflow}{%
\section{Workflow}\label{workflow}}

\hypertarget{installation}{%
\subsection{Installation}\label{installation}}

\pkg{pksensi} is available from the comprehensive R Archive Network at
\url{https://CRAN.R-project.org/package=pksensi}. The latest up-to-date
version is available through GitHub at
\url{https://github.com/nanhung/pksensi}.

\hypertarget{parameter-matrix-generation}{%
\subsection{Parameter matrix
generation}\label{parameter-matrix-generation}}

We adopted eFAST (extended Fourier Amplitude Sensitivity Test), a widely
used global sensitivity analysis approach in biomathematical modeling,
especially for ordinary differential equation-based dynamical models, in
this package (Saltelli et al., 1999). Most of the available eFAST
functions, such as the algorithm to generate the parameter space and to
set the sampling frequency, were sourced from the R \pkg{sensitivity}
package (Pujol et al., 2017). To test the convergence and robustness of
the sensitivity measurement, we included a random phase-shift approach
to replicate sampling from random starting points across parameter
space. This can be written as:

\[ x_i = \frac{1}{2} + \frac{1}{\pi}\arcsin(\sin(\omega_is + \varphi_i)) \]

where \(x_i\) is the nominal value of the i-th parameter, and
\(\omega_i\) is a vector giving the set of frequencies, one frequency
for each parameter. The default set of frequencies is based on the study
from Saltelli et al.~(1999). The first frequency of the vector
\(\omega\) is assigned to each factor \(x_i\) in turn (corresponding to
the estimation of Sobol indices of first and total order, respectively).
Through the random phase-shift, the robust result of the sensitivity
measurement should be similar across replications under the same sample
size.

To investigate the convergence of sensitivity indices, we adopted the
exam approach proposed by Sarrazin et al.~(2016). This method
quantitatively assesses convergence by computing the width of 95\%
confidence intervals for all parameters across all time-points and
output variables. To integrate the random phase-shift with eFAST, we
built the \proglang{rfast99} function, which can be used to perform the
sensitivity test for the whole pharmacokinetic modeling process. The
function \proglang{rfast99} in pksensi can sample and generate the
testing parameter matrix based on a given argument of sample size n and
probability distribution (e.g., mean/s.d. of normal distribution,
meanlog/sdlog of log-normal distribution, and minimum/maximum of uniform
distribution). The number of model evaluations is equal to the sample
size times the number of model parameters.

\hypertarget{pharmacokinetic-modeling}{%
\subsection{Pharmacokinetic modeling}\label{pharmacokinetic-modeling}}

The \pkg{pksensi} provides a useful function to solve ordinary
differential equations in pharmacokinetic models using analytical and
numerical approaches. The solution can be performed under the pure R
coding environment by linking pksensi with the deSolve package (Soetaert
et al., 2018). However, the most efficient way to achieve high
computational speed is through compiled, lower-level languages, such as
FORTRAN, C, or C++. The \pkg{pksensi} includes a function to compile and
create dynamic-link libraries (.dll) on Windows and shared objects (.so)
on Unix-liked systems (e.g., Linux and MacOS). This compiled file can
load and execute through the build-in function in R. However, Windows
users need to install Rtools or MinGW to compile the source code by
using the GNU GCC compiler. More requirements are detailed in the study
from Soetaert et al.~(2010). The C implementation of the example model
can be found as an example within MCSim v6.0.1 (Bois, 2009). The
\pkg{pksensi} can also link with MCSim to compile the model code, used
in solving each system of equations.

\hypertarget{output-visualization-and-decision-making}{%
\subsection{Output visualization and decision
making}\label{output-visualization-and-decision-making}}

The output of sensitivity analysis returns a list that contains the
given time-points and values of all state variables (e.g., chemical
concentrations in blood). This format is particularly suited for further
graphical routines within pkseni. Also, a plot can can create
time-dependent sensitivity measurements of first and total order effect
for all parameters in one figure. Furthermore, pksensi includes
functions to check the convergence and sensitivity of model parameters,
providing a means to assess the robustness of the sensitivity
measurement. We also developed a ``cut-off''-based approach in this
package to accurately distinguish between ``influential'' and
``non-influential'' parameters. Finally, pksensi includes numerous
visualization tools for the effective investigation and communication of
results in decision making.

\hypertarget{examples}{%
\section{Examples}\label{examples}}

\hypertarget{one-compartment-taxicokinetic-model}{%
\subsection{One-compartment taxicokinetic
model}\label{one-compartment-taxicokinetic-model}}

\hypertarget{equations}{%
\subsubsection{Equations}\label{equations}}

In this example, We use a simple, one-compartment PK model from
\texttt{httk} package \citep{JSSv079i04} to demonstrate how
\texttt{pksensi} can be applied to pharmacokinetic studies. The
differential equations for the one-compartment pharmacokinetic model can
be written as:

\[\frac{dA_{gutlumen}}{dt} = -k_{gutabs} \cdot A_{gutlumen} + g(t)\]
\[\frac{dC_{rest}}{dt} = \frac{k_{gutabs}}{V_{dist}}-k_{elim} \cdot C_{rest}\]

where \(A_{gutlumen}\) is the state variable that describes the quantity
of compound in gut lumen (mol), \(k_{gutabs}\) is the absorption rate
constant that describes the chemical absorption from the gut lumen into
gut tissue through first-order processes (/h), \(V_{dist}\) is the
volume of distribution (L), and \(k_{elim}\) is the elimination rate
constant (/h), which is equal to the total clearance divided by the
volume of distribution. The time-dependent function \(g(t)\) is used to
describe the oral dosing schedule. \(C_{rest}\) is the chemical
concentration in plasma that can be used to compare with observed
results in a pharmacokinetic experiment (mol/L).

\hypertarget{model-implementations-with-r-desolve-package}{%
\subsubsection{Model implementations with R deSolve
package}\label{model-implementations-with-r-desolve-package}}

In the beginning, we need to pre-install GNU MCSim \citep{JSSv002i09}
and related compiler to let us generate .c file and executable file. The
GNU MCSim can be installed by following the instruction in GNU MCSim's
manual on \url{https://www.gnu.org/software/mcsim/mcsim.html} or using
the build-in function \texttt{mcsim\_install()} in \texttt{pksensi}. The
GNU compiler is necessary for users that use Linux or MacOS. For Windows
users, you should install Rtools on
\url{https://cran.r-project.org/bin/windows/Rtools/} and use
\texttt{Sys.setenv()} to set the working path of compiler. The
\texttt{Sys.which("gcc")} and
\texttt{system(\textquotesingle{}g++\ -v\textquotesingle{})} can check
whether we can run compiler correctly.

We first implemented this model in R by compiling the file written in C.
pksensi allows users to select the preferred method to solve the
pharmacokinetic model, either with the \texttt{deSolve} package or with
GNU MCSim through the compile function. However, running model under GNU
MCSim native code can have faster speed to obtain the model outputs.

The following R script can download and compile the model description
file (\texttt{pbtk1cpt.model}) and use \texttt{deSolve} package
\citep{JSSv033i09} to solve ordinary differential equations in our
model. The example model code of one-compartment PBTK model is available
with pksensi package:

\begin{CodeChunk}

\begin{CodeInput}
R> pbtk1cpt_model()
R> cat(readLines("pbtk1cpt.model"), sep = "\n")
\end{CodeInput}

\begin{CodeOutput}
#> # ------------------------------------------
#> # pbtk1cpt.model (Based on R httk package)
#> # ------------------------------------------
#> 
#> #
#> States  = { Agutlument, Acompartment, Ametabolized, AUC};
#> 
#> #
#> Outputs = {Ccompartment};
#> 
#> # Parameters
#> vdist = 0;
#> ke = 0;
#> kgutabs = 1;
#> 
#> #
#> Dynamics {
#>   Ccompartment = Acompartment / vdist;
#>   dt (Agutlument)  = - kgutabs * Agutlument;
#>   dt (Acompartment)  = kgutabs * Agutlument - ke * Acompartment;
#>   dt (Ametabolized) = ke * Acompartment;
#>   dt (AUC) = Ccompartment;
#> }
#> 
#> End.
\end{CodeOutput}
\end{CodeChunk}

Then, use \texttt{compile\_model()} to generate the executable files
(\texttt{pbtk1cpt.dll} on Windows or \texttt{pbtk1cpt.so} on other
systems) and R file (\texttt{pbtk1cpt\_inits.R}) with default input
parameters and initial state settings with the definition of
\texttt{application\ =\ "R"}.

\begin{CodeChunk}

\begin{CodeInput}
R> mName <- "pbtk1cpt"
R> compile_model(mName, application = "R")
\end{CodeInput}

\begin{CodeOutput}
#> * Created file 'pbtk1cpt.so'.
\end{CodeOutput}

\begin{CodeInput}
R> source(paste0(mName, "_inits.R"))
\end{CodeInput}
\end{CodeChunk}

The parameter values and initial states can be customized to specify the
properties and schedule for the given dosing scenario.

\begin{CodeChunk}

\begin{CodeInput}
R> parms <- initParms()
R> parms["vdist"] <- 0.74
R> parms["ke"] <- 0.28
R> parms["kgutabs"] <- 2.18
R> initState <- initStates(parms=parms)
R> initState["Agutlument"] <- 10
\end{CodeInput}
\end{CodeChunk}

\begin{CodeChunk}

\begin{CodeInput}
R> parms
\end{CodeInput}

\begin{CodeOutput}
#>   vdist      ke kgutabs 
#>    0.74    0.28    2.18
\end{CodeOutput}
\end{CodeChunk}

\begin{CodeChunk}

\begin{CodeInput}
R> initState
\end{CodeInput}

\begin{CodeOutput}
#>   Agutlument Acompartment Ametabolized          AUC 
#>           10            0            0            0
\end{CodeOutput}
\end{CodeChunk}

\begin{CodeChunk}

\begin{CodeInput}
R> Outputs
\end{CodeInput}

\begin{CodeOutput}
#> [1] "Ccompartment"
\end{CodeOutput}
\end{CodeChunk}

In the current setting, we assumed the initial condition of the intake
chemical to be 10 mol. The \texttt{initParms} and \texttt{initStates}
functions were used to customize the parameter values and the initial
state that will be used in the \texttt{solve\_fun} function. These
parameter value can be adopted from the \texttt{httk} package, which
includes physico-chemical and drug biological properties for 553
chemicals. In this case, we used the parameter value of theophylline in
this example. The given \texttt{vdist}, \texttt{ke}, and
\texttt{kgutabs} are 0.74, 0.28, and 2.18, respectively.

Through \texttt{ode} function in \texttt{deSolve} package, we can
visualize the pharmacokinetic according to the given parameter
conditions such as time points (\texttt{times}):

\begin{CodeChunk}

\begin{CodeInput}
R> times <- seq(from = 0.01, to = 24.01, by = 1)
R> y <- deSolve::ode(initState, times, func = "derivs", parms = parms, 
R+                   dllname = mName, initfunc = "initmod", nout = 1, outnames = Outputs)
\end{CodeInput}
\end{CodeChunk}

\begin{CodeChunk}

\begin{CodeInput}
R> plot(y)
\end{CodeInput}
\begin{figure}

{\centering \includegraphics{manuscript_files/figure-latex/unnamed-chunk-8-1} 

}

\caption[Figure 1]{Figure 1. Simulation results of one-compartment PBTK model.}\label{fig:unnamed-chunk-8}
\end{figure}
\end{CodeChunk}

To conduct sensitivity analysis for the parameters in one-compartment
pharmacokinetic model in this case, we want to quantify the impact of
these three parameters on the chemical concentration in plasma during
24-hour time period post intake. We assume a uniform distribution for
the estimate for each parameter with the coefficient of uncertainty
within 50\%. The parameter ranges are assumed to be (0.37, 1.12) for
\texttt{vdist}, (0.0058, 0.0174) for \texttt{ke}, and (0.045, 0.136) for
\texttt{kgutabs}. The sample number determines the robustness of the
result of sensitivity analysis. Higher sample numbers can generate
narrow confidence intervals for sensitivity measurements across
different replications. However, they might cause heavy computational
burden for complex models. Here we use a sample number of 400 with 20
replications:

\begin{CodeChunk}

\begin{CodeInput}
R> LL <- 0.5 
R> UL <- 1.5
R> q <- "qunif"
R> q.arg <- list(list(min = parms["vdist"] * LL, max = parms["vdist"] * UL),
R+              list(min = parms["ke"] * LL, max = parms["ke"] * UL), 
R+              list(min = parms["kgutabs"] * LL, max = parms["kgutabs"] * UL)) 
R> set.seed(1234)
R> params <- c("vdist", "ke", "kgutabs")
R> x <- rfast99(params, n = 200, q = q, q.arg = q.arg, replicate = 20)
\end{CodeInput}
\end{CodeChunk}

Because the pharmacokinetic model is being used to describe a continuous
process for the chemical concentration over time, the sensitivity
measurements can also show the time-dependent relationships for each
model parameter. Here we define the output time points to examine the
change of the parameter sensitivity over time. To solve the
pharmacokinetic model through deSolve, we need to provide the details of
the argument:

\begin{CodeChunk}

\begin{CodeInput}
R> y <- solve_fun(x, times, initState = initState,
R+                outnames = Outputs, dllname = mName)
R> tell2(x,y)
\end{CodeInput}
\end{CodeChunk}

To create the time-dependent sensitivity measurement, we set the time
duration from 0.01 to 24.01 hours in the example. The
\texttt{initParmsfun} is used to generate the sampling value for each
parameter. The \texttt{outnames}, \texttt{dllname}, \texttt{func},
\texttt{initfunc} are based on the arguments from the ode function in
\texttt{deSolve} package. The details of the model structure and these
arguments are defined in \texttt{pbtk1comp.c}. and
\texttt{pbtk1comp\_inits.R}. Finally, the \texttt{tell2} function is
used to integrate the parameter values and the output results of
numerical analysis that were generated and stored in variables x and y.
The result of object x is an object of rfast99, which has specific
\texttt{print}, \texttt{plot}, and \texttt{check} method. The print
function gives the sensitivity and convergence indices for main,
interaction, and total order at each time point. In addition to print
out the result of sensitivity analysis, the more efficient way to
distinguish the influence of model parameter is to visualize them. The
time-dependent sensitivity indices are shown in Figure 2.

\begin{CodeChunk}

\begin{CodeInput}
R> plot(x)
\end{CodeInput}
\begin{figure}

{\centering \includegraphics{manuscript_files/figure-latex/unnamed-chunk-11-1} 

}

\caption[Figure 2]{Figure 2. Time-dependent sensitivity indices of the plasma concentration estimated from one-compartment PBTK model during 24 hour time period intake.}\label{fig:unnamed-chunk-11}
\end{figure}
\end{CodeChunk}

Here, we can find that \texttt{vdist} and \texttt{ke} are dominating the
plasma concentration in the before and after 5-hour post chemical
intake, respectively, representing that the elimination is a key
parameter to dominate the plasma concentration. Besides, the
\texttt{kgutabs} only plays a crucial role to determine the plasma
concentration in the first hour. The relationship between concentration
and the parameters can be plotted as follow (Figure 3):

\begin{CodeChunk}

\begin{CodeInput}
R> par(mfrow = c(3,3), mar = c(2,2,2,2), oma = c(2,2,1,1))
R> plot(x$a[,1,"vdist"], y[,1,"0.01",], main = "vdist")
R> text(1, .7, "t=0.01",cex = 1.2)
R> plot(x$a[,1,"ke"], y[,1,"0.01",], main = "ke")
R> plot(x$a[,1,"kgutabs"], y[,1,"0.01",], main = "kgutabs")
R> plot(x$a[,1,"vdist"], y[,1,"2.01",])
R> text(1, 18, "t=2.01",cex = 1.2)
R> plot(x$a[,1,"ke"], y[,1,"2.01",])
R> plot(x$a[,1,"kgutabs"], y[,1,"2.01",])
R> plot(x$a[,1,"vdist"], y[,1,"24.01",])
R> text(1, .7, "t=24.01",cex = 1.2)
R> plot(x$a[,1,"ke"], y[,1,"24.01",])
R> plot(x$a[,1,"kgutabs"], y[,1,"24.01",])
R> mtext("parameter", SOUTH<-1, line=0.4, outer=TRUE)
R> mtext("Ccompartment", WEST<-2, line=0.4, outer=TRUE)
\end{CodeInput}
\begin{figure}

{\centering \includegraphics{manuscript_files/figure-latex/unnamed-chunk-12-1} 

}

\caption[Figure 3]{Figure 3. The relationship between model parameter and estimated concentration under the time-point of 0.01, 2.01, and 24.01 hr}\label{fig:unnamed-chunk-12}
\end{figure}
\end{CodeChunk}

The x is a list of class ``rfast99'', containing all the input arguments
detailed before and the calculated sensitivity indices of first order
(\texttt{mSI}), interaction (\texttt{iSI}), and total order
(\texttt{tSI}). The convergence indices are also stored in the list
named \texttt{mCI}, \texttt{iCI}, and \texttt{tCI}. The parameter values
are stored in an array \texttt{x\$a} with c(model evaluation,
replication, parameters).

\begin{CodeChunk}

\begin{CodeInput}
R> dim(x$a)
\end{CodeInput}

\begin{CodeOutput}
#> [1] 600  20   3
\end{CodeOutput}
\end{CodeChunk}

In addition, the output are also formated with c(model evaluation,
replication, time, variable).

\begin{CodeChunk}

\begin{CodeInput}
R> dim(y)
\end{CodeInput}

\begin{CodeOutput}
#> [1] 600  20  25   1
\end{CodeOutput}
\end{CodeChunk}

The \texttt{check()} is a useful function to determine which parameters
have relative lower sensitivity measurement across the given time
interval, and therefore can be applied parameter fixing in model
calibration. The argument of \texttt{SI.cutoff} is setting at 0.5 to
detect the relative non-influential parameters in this case.

\begin{CodeChunk}

\begin{CodeInput}
R> check(x, SI.cutoff = 0.5)
\end{CodeInput}

\begin{CodeOutput}
#> 
#> Sensitivity check ( Index > 0.5 )
#> ----------------------------------
#> First order:
#>  vdist ke 
#> 
#> Interaction:
#>   
#> 
#> Total order:
#>  vdist ke 
#> 
#> Unselected factors in total order:
#>  kgutabs 
#> 
#> 
#> Convergence check ( Index > 0.05 )
#> ----------------------------------
#> First order:
#>   
#> 
#> Interaction:
#>   
#> 
#> Total order:
#> 
\end{CodeOutput}
\end{CodeChunk}

Based on the sensitivity measurement of the total order, the result
shows that \texttt{kgutabs} has relative lower measurement of
sensitivity index.

\hypertarget{model-implementations-with-gnu-mcsim}{%
\subsubsection{Model implementations with GNU
MCSim}\label{model-implementations-with-gnu-mcsim}}

Alternatively, to solve ODE by using GNU MCSim, we need to change the
argument to \texttt{application\ =\ mcsim} in \texttt{compile\_model()}.
Rather than apply R \texttt{deSolve} to solve differential equations,
the GNU MCSim can provide higher computational speed in global
sensitivity analysis.

\begin{CodeChunk}

\begin{CodeInput}
R> system.time(y<-solve_fun(x, times, initState = initState, 
R+                          outnames = Outputs, dllname = mName))
\end{CodeInput}

\begin{CodeOutput}
#>    user  system elapsed 
#>   5.656   0.019   5.695
\end{CodeOutput}
\end{CodeChunk}

\begin{CodeChunk}

\begin{CodeInput}
R> compile_model(mName, application = "mcsim")
\end{CodeInput}

\begin{CodeOutput}
#> * Created executable file 'mcsim.pbtk1cpt'.
\end{CodeOutput}
\end{CodeChunk}

Sililiar to \texttt{solve\_fun} that can define the initial parameter
and state values through input function, the \texttt{solve\_mcsim} has a
\texttt{condition} argument that is used to givien the specific input
value such as oral dose or fixing parameter value or initial state
variable.

\begin{CodeChunk}

\begin{CodeInput}
R> conditions <- c("Agutlument = 10") # Set the initial state of Agutlument = 10 
R> system.time(y<-solve_mcsim(x, mName = mName, 
R+                            params = params,
R+                            vars = Outputs,
R+                            time = times,
R+                            condition = conditions))
\end{CodeInput}

\begin{CodeOutput}
#> * Created input file "input.in".
\end{CodeOutput}

\begin{CodeOutput}
#>    user  system elapsed 
#>   1.604   0.084   1.523
\end{CodeOutput}
\end{CodeChunk}

Under the same given condition, it takes 5-6 (\texttt{deSolve}) and 1-2
(GNU MCSim) seconds to solve model. The \texttt{solve\_mcsim()} shows
the better computational performance than \texttt{solve\_fun()} in
\texttt{pksensi}.

\hypertarget{acetaminophen-pbpk-model}{%
\subsection{Acetaminophen-PBPK model}\label{acetaminophen-pbpk-model}}

\hypertarget{uncertainty-and-sensitivity-analysis}{%
\subsubsection{Uncertainty and sensitivity
analysis}\label{uncertainty-and-sensitivity-analysis}}

The aim of this section is to reproduce our previous published
\citep{fphar201800588} result of global sensitivity analysis for
acetaminophen PBPK model through \texttt{pksensi}. The model codes are
included in this package and can be generated through
\texttt{pbpk\_apap\_model()}. We applied the global sensitivity analysis
workflow to the original published model with 21 model parameters
\citep{s13318-015-0253-x}. The descriptions of each parameter and the
sampling ranges are list in Table 1.

\textless{}Table 1\textgreater{}

Same as the example of one-compartment PBTK model. The model parameter
and the corresponding sampling range should be defined to create the
parameter matrix. Previously, the probability distributions of model
parameters were set to either truncated normal or uniform distribution
when the parameters have informative prior information or not. To
rapidly reach the acceptance convergence, we apply uniform distribution
for all testing parameters. The ranges of informative parameters are set
to 1.96-times difference for single side (approximate 54.6 times
difference between minimum and maximum) under log-scaled. The nominal
values of informative model parameters were defined as:

\begin{CodeChunk}

\begin{CodeInput}
R> # Nominal value
R> Tg <- log(0.23)
R> Tp <- log(0.033)
R> CYP_Km <- log(130)
R> SULT_Km_apap <- log(300)
R> SULT_Ki <- log(526)
R> SULT_Km_paps <- log(0.5)
R> UGT_Km <- log(6.0e3)
R> UGT_Ki <- log(5.8e4)
R> UGT_Km_GA <-log(0.5)
R> Km_AG <- log(1.99e4)
R> Km_AS <- log(2.29e4)
R> 
R> rng <- 1.96 
\end{CodeInput}
\end{CodeChunk}

Generally, The wide range of parameter value might cause the
computational error in the solver. One of the effective ways to prevent
this problem is to adjust the value of relative and absolute error
tolerance to control the error appearance by resetting these parameters
in a lower value. The \texttt{generate\_infile()} provide the arguments
of \texttt{rtol} and \texttt{atol} that can be adjusted to prevent the
unwanted error. However, the modification will slow down the
computational speed. Therefore, the alternative method to prevent the
computational error is to detect the crucial parameter range that causes
the problem. Also, setting the maximum number of (internally defined)
steps to higher value instead of using the default value (500) can
prevent this problem. The maximum number of step is set to 5000 in this
case.

In this test case, we adjusted the range of \texttt{SULT\_VmaxC} and
\texttt{UGT\_VmaxC} from U(0, 15) to U(0, 10). The relative and absolute
error tolerance were set to 1e-7 and 1e-9, respectively, to prevent the
computational error in MCSim,

\begin{CodeChunk}

\begin{CodeInput}
R> params <- c("lnTg", "lnTp", "lnCYP_Km","lnCYP_VmaxC",
R+            "lnSULT_Km_apap","lnSULT_Ki","lnSULT_Km_paps","lnSULT_VmaxC",
R+            "lnUGT_Km","lnUGT_Ki","lnUGT_Km_GA","lnUGT_VmaxC",
R+            "lnKm_AG","lnVmax_AG","lnKm_AS","lnVmax_AS",
R+            "lnkGA_syn","lnkPAPS_syn", "lnCLC_APAP","lnCLC_AG","lnCLC_AS")
R> q <- "qunif"
R> q.arg <-list(list(Tg-rng, Tg+rng),
R+              list(Tp-rng, Tp+rng),
R+              list(CYP_Km-rng, CYP_Km+rng),
R+              list(-2., 5.),
R+              list(SULT_Km_apap-rng, SULT_Km_apap+rng),
R+              list(SULT_Ki-rng, SULT_Ki+rng),
R+              list(SULT_Km_paps-rng, SULT_Km_paps+rng),
R+              list(0, 10),
R+              list(UGT_Km-rng, UGT_Km+rng),
R+              list(UGT_Ki-rng, UGT_Ki+rng),
R+              list(UGT_Km_GA-rng, UGT_Km_GA+rng),
R+              list(0, 10),
R+              list(Km_AG-rng, Km_AG+rng),
R+              list(7., 15),
R+              list(Km_AS-rng, Km_AS+rng),
R+              list(7., 15),
R+              list(0., 13),
R+              list(0., 13),
R+              list(-6., 1),
R+              list(-6., 1),
R+              list(-6., 1))
R> 
R> times <- seq(from = 0.1, to = 12.1, by = 0.2)
R> set.seed(1234)
R> x <- rfast99(params = params, n = 512, q = q, q.arg = q.arg, replicate = 10) 
\end{CodeInput}
\end{CodeChunk}

After creating the \texttt{pbpk\_apap.model} in the working directory,
the next step is to generate the executable files
(\texttt{mcsim.pbpk\_apap}) through \texttt{compile\_model()}.

\begin{CodeChunk}

\begin{CodeInput}
R> mName <- "pbpk_apap"
R> pbpk_apap_model()
R> compile_model(mName, application = "mcsim")
\end{CodeInput}

\begin{CodeOutput}
#> * Created executable file 'mcsim.pbpk_apap'.
\end{CodeOutput}
\end{CodeChunk}

To improve the computational speed, this case only uses MCSim to
estimate the concentration of acetaminophen (APAP) and its metabolites
glucuronide (AG) and sulfate (AS) in plasma. The setting oral dose of
APAP is 20 mg/kg in this example. Generally, the input dosing method can
be defined through the \texttt{condition} argument. Since the unit of
the given dose is mg/kg, the \texttt{mgkg\_flag} is set to 1 to declare
the statement. More definition of input can be found in the section of
input functions in GNU MCSim User's Manual
(\url{https://www.gnu.org/software/mcsim/mcsim.html\#Input-functions}).

\begin{CodeChunk}

\begin{CodeInput}
R> vars <- c("lnCPL_APAP_mcgL", "lnCPL_AG_mcgL", "lnCPL_AS_mcgL")
R> conditions <- c("mgkg_flag = 1",
R+                 "OralExp_APAP = NDoses(2, 1, 0, 0, 0.001)",
R+                 "OralDose_APAP_mgkg = 20.0")
R> generate_infile(params = params,
R+                 vars = vars,
R+                 time = times, 
R+                 condition = conditions,
R+                 rtol = 1e-7, atol = 1e-9)
\end{CodeInput}

\begin{CodeOutput}
#> * Created input file "input.in".
\end{CodeOutput}

\begin{CodeInput}
R> system.time(y <- solve_mcsim(x, mName = mName, 
R+                              params = params,
R+                              vars = vars,
R+                              time = times,
R+                              condition = conditions,
R+                              generate.infile = F))
\end{CodeInput}

\begin{CodeOutput}
#>    user  system elapsed 
#> 150.905   1.794 152.061
\end{CodeOutput}

\begin{CodeInput}
R> tell2(x,y)
\end{CodeInput}
\end{CodeChunk}

The plotting function can output the result of time-dependent
sensitivity measurement to determine the parameter impact on model
output over time (Figure 1).

\begin{CodeChunk}

\begin{CodeInput}
R> plot(x, vars = "lnCPL_AG_mcgL")
\end{CodeInput}
\begin{figure}

{\centering \includegraphics{manuscript_files/figure-latex/unnamed-chunk-23-1} 

}

\caption[Figure 4]{Figure 4. }\label{fig:unnamed-chunk-23}
\end{figure}
\end{CodeChunk}

The uncertainty analysis is a crucial step before model calibration. We
can apply uncertainty analysis through the \texttt{pksim()} by the given
name of output variables (Figure 2). Through this visualization
approach, we can recognize whether the simulated outputs can accurately
simulate the same concentration profile as the in-vivo experiment under
the setting of parameter ranges. In Figure 2, all experiment points are
included in the intervals, representing the acceptable set of the
parameter range.

\begin{CodeChunk}

\begin{CodeInput}
R> par(mfrow = c(2,2), mar = c(2,2,1,1), oma = c(2,2,1,1))
R> pksim(y, vars = "lnCPL_APAP_mcgL")
R> text(1, 15, "APAP",cex = 1.2)
R> points(APAP$Time, log(APAP$APAP * 1000))
R> pksim(y, vars = "lnCPL_AG_mcgL", legend = F)
R> text(1, 15, "AG",cex = 1.2)
R> points(APAP$Time, log(APAP$AG * 1000))
R> pksim(y, vars = "lnCPL_AS_mcgL", legend = F)
R> text(1, 15, "AS",cex = 1.2)
R> points(APAP$Time, log(APAP$AS * 1000))
R> mtext("Time", SOUTH<-1, line=0.4, cex=1.2, outer=TRUE)
R> mtext("Conc.", WEST<-2, line=0.4, cex=1.2, outer=TRUE)
\end{CodeInput}
\begin{figure}

{\centering \includegraphics{manuscript_files/figure-latex/unnamed-chunk-24-1} 

}

\caption[Figure 5]{Figure 5. }\label{fig:unnamed-chunk-24}
\end{figure}
\end{CodeChunk}

In addition, through using the \texttt{check()}, the parameter with
sensitivity and convergence indices over the given condition can be
easily detected for all output variables.

\begin{CodeChunk}

\begin{CodeInput}
R> check(x)
\end{CodeInput}

\begin{CodeOutput}
#> 
#> Sensitivity check ( Index > 0.05 )
#> ----------------------------------
#> First order:
#>  lnTg lnTp lnSULT_VmaxC lnUGT_Km lnUGT_VmaxC lnKm_AG lnVmax_AG lnKm_AS lnVmax_AS lnkGA_syn lnkPAPS_syn lnCLC_APAP lnCLC_AG lnCLC_AS 
#> 
#> Interaction:
#>  lnTg lnSULT_Km_apap lnSULT_VmaxC lnUGT_VmaxC lnVmax_AG lnVmax_AS lnkGA_syn lnkPAPS_syn lnCLC_APAP lnCLC_AG lnCLC_AS 
#> 
#> Total order:
#>  lnTg lnTp lnSULT_Km_apap lnSULT_VmaxC lnUGT_Km lnUGT_VmaxC lnKm_AG lnVmax_AG lnKm_AS lnVmax_AS lnkGA_syn lnkPAPS_syn lnCLC_APAP lnCLC_AG lnCLC_AS 
#> 
#> Unselected factors in total order:
#>  lnCYP_Km lnCYP_VmaxC lnSULT_Ki lnSULT_Km_paps lnUGT_Ki lnUGT_Km_GA 
#> 
#> 
#> Convergence check ( Index > 0.05 )
#> ----------------------------------
#> First order:
#>  lnTg lnTp lnSULT_VmaxC lnUGT_Km lnUGT_VmaxC lnVmax_AG lnKm_AS lnVmax_AS lnkGA_syn lnkPAPS_syn lnCLC_APAP lnCLC_AG lnCLC_AS 
#> 
#> Interaction:
#>  lnSULT_Km_apap lnSULT_VmaxC lnUGT_VmaxC lnKm_AG lnVmax_AG lnKm_AS lnVmax_AS lnkGA_syn lnkPAPS_syn lnCLC_APAP lnCLC_AG lnCLC_AS 
#> 
#> Total order:
#>  lnTg lnTp lnSULT_Km_apap lnSULT_VmaxC lnUGT_Km lnUGT_VmaxC lnKm_AG lnVmax_AG lnKm_AS lnVmax_AS lnkGA_syn lnkPAPS_syn lnCLC_APAP lnCLC_AG lnCLC_AS
\end{CodeOutput}
\end{CodeChunk}

The \texttt{check()} also provides some feasible argument to specify the
target output or change the cut-off value.

\begin{CodeChunk}

\begin{CodeInput}
R> check(x, vars = "lnCPL_APAP_mcgL", SI.cutoff = 0.1, CI.cutoff = 0.1)
\end{CodeInput}

\begin{CodeOutput}
#> 
#> Sensitivity check ( Index > 0.1 )
#> ----------------------------------
#> First order:
#>  lnTg lnTp lnSULT_VmaxC lnCLC_APAP 
#> 
#> Interaction:
#>  lnSULT_VmaxC lnkPAPS_syn 
#> 
#> Total order:
#>  lnTg lnTp lnSULT_VmaxC lnkPAPS_syn lnCLC_APAP 
#> 
#> Unselected factors in total order:
#>  lnCYP_Km lnCYP_VmaxC lnSULT_Km_apap lnSULT_Ki lnSULT_Km_paps lnUGT_Km lnUGT_Ki lnUGT_Km_GA lnUGT_VmaxC lnKm_AG lnVmax_AG lnKm_AS lnVmax_AS lnkGA_syn lnCLC_AG lnCLC_AS 
#> 
#> 
#> Convergence check ( Index > 0.1 )
#> ----------------------------------
#> First order:
#>  lnTg lnCLC_APAP 
#> 
#> Interaction:
#>  lnkPAPS_syn 
#> 
#> Total order:
#>  lnTg lnSULT_VmaxC lnkPAPS_syn lnCLC_APAP
\end{CodeOutput}
\end{CodeChunk}

\hypertarget{heatmap-visualization-combined-with-an-index-cut-off}{%
\subsubsection{Heatmap visualization combined with an index
``cut-off''}\label{heatmap-visualization-combined-with-an-index-cut-off}}

Based on our previous study, we proposed the heatmap visualization
approach to distinguish ``influential'' and ``non-influential''
parameters with a cut-off. Through the given argument \texttt{order}, we
can select the specific order of sensitivity measurement that we're
interested in (Figure 3 \& 4).

\begin{CodeChunk}

\begin{CodeInput}
R> heat_check(x, order = "interaction")
\end{CodeInput}


\begin{center}\includegraphics{manuscript_files/figure-latex/unnamed-chunk-27-1} \end{center}

\end{CodeChunk}

\begin{CodeChunk}

\begin{CodeInput}
R> heat_check(x, order = "total order")
\end{CodeInput}


\begin{center}\includegraphics{manuscript_files/figure-latex/unnamed-chunk-28-1} \end{center}

\end{CodeChunk}

Also, adding the \texttt{index\ =\ "CI"} in the function can further
investigate the convergence of the sensitivity index. Based on the
current setting of sampling number, most parameters cannot reach the
acceptable criteria of convergence. Therefore, the higher number of
sampling is necessary.

\begin{CodeChunk}

\begin{CodeInput}
R> heat_check(x, index = "CI", CI.cutoff = 0.05)
\end{CodeInput}


\begin{center}\includegraphics{manuscript_files/figure-latex/unnamed-chunk-29-1} \end{center}

\end{CodeChunk}

\hypertarget{code-formatting}{%
\subsection{Code formatting}\label{code-formatting}}

Don't use markdown, instead use the more precise latex commands:

\begin{itemize}
\item
  \proglang{Java}
\item
  \pkg{plyr}
\item
  \code{print("abc")}
\end{itemize}

\hypertarget{concluding-remarks}{%
\section{Concluding remarks}\label{concluding-remarks}}

\hypertarget{acknowledgments}{%
\section{Acknowledgments}\label{acknowledgments}}

This work was funded, in part, by grant 1U01FD005838 from the U.S. FDA.
This abstract reflects the views of the author and should not be
construed to represent FDA's views or policies. We thank Dr.~Yasha
Hartberg and Dr.~Barbara Gastel at the Texas A\&M University for
reviewing the manuscript and consultation.



\end{document}

